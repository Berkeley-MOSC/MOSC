\documentclass[10pt]{article}
\setlength\parindent{0pt}
\usepackage[margin=1in]{geometry}
\begin{document}
\title{\textsc{IEOR 185: Solution Proposal}}
\author{
    Gavin Chan,
    Michael Deng,
    Nicholas Fong,
    Calvin Lui,
    Mei Wan\\
    mosc[at]lists.berkeley.edu\\
}
\date{18 February 2016}
\maketitle

\section{Overview and Problem Statement}
\subsection*{Value Proposition}
During emergencies, phone networks often become unavailable with the influx of
calls.  This is because the phone infrastructure cannot support the influx of
connections, as it is designed to be resilient under common use, where a
minority of users are placing calls and sending texts at a given time.  By
multiplexing calls, network capacity can be increased, ensuring that civilians
and first-responders alike can stay in touch and effectively coordinate
response.
\subsection*{Target Market}
Telcos, local and federal emergency response agencies, municipal governments of
disaster prone regions around the globe.
\subsection*{Beneficiaries}
Civilians, emergency responders and supporting staff, aid organizations,
extended (out-of-area) families and friends, and others that communicate within
and in to or out of an area affected by an large-scale emergency.
\subsection*{Market Alternatives}
\subsubsection*{goTenna\footnote{Snippet taken from Nick's ecosystem writeup}}
The premise behind the goTenna is simple: a user with a goTenna connects their
phone to the goTenna via BTLE, and then uses a proprietary application to send
texts or a location to another goTenna user.   The goTenna has a range of about
a mile in urban areas, and up to four miles in open areas.  The goTenna is
marketed toward those that want to be able to communicate off-the-grid; goTenna
calls itself ``adventure-ready" -- combined with the images of backpacking and
the wilderness -- indicating its target users are adventure-seekers whose
travels often take them away from cell phone towers.

\subsubsection*{Globalstar}
Globalstar is a satellite startup that was negotiating with the FCC about
licensing open frequencies in emergencies that close to frequencies they already
own so that they can provide communications for on-the-ground personnel.
Unfortunately, it doesn't seem like there's any information about whether or not
Globalstar was able to acquire the rights to these open frequencies.

\subsubsection*{}

\section{MVP Solution Recommendation}
TODO

\subsection*{User Stories}
\begin{enumerate}
    \item After working for an extended period of time in an area devastated by
        an earthquake, a deployed USAR worker's radio dies.  He finds a group of
        Immediates in a collapsed building, and instead of having to run back to
        the command center to get help because the phone networks are down, he
        can call the IC and get help immediately
    \item After an earthquake, a mother is able to call her son's school and
        confirm that her son is safe.  She is able to place a call to her
        husband and learns that a large section of freeway overpass has
        collapsed and that he cannot make it to their son's school to pick him
        up, and after placing more calls to a few of her neighbors, goes to the
        school to pick up her son an her neighbors' children, leading to less
        congested roads and a stronger sense of community.
    \item While vacationing, a young couple learn that copious amounts of rain
        has led to flash flooding in their hometown.  Fortunately, their parents
        are able to call them to let them know that they are safe at home and,
        although their home lost power, they have ample food, clean water, and
        candles to last through the storm.
\end{enumerate}
\subsection*{Mockups}
TODO Nick

\subsection*{Future Roadmap}
While this solution can undoubtedly help increase phone network capacity, it
doesn't help in the event of widespread infrastructure outages -- that is, when
trunk lines are severed or the CO has burnt to the ground.  Implementing
a solution that would enable RF communication without relying on infrastructure
would be a great next step, but is unfortunately out of the scope of what can be
accomplished in one semester.

\section{Implementation}
\subsection*{KPIs}
Using the simulation that we create to cause an influx of calls, we can
determine how many concurrent phone calls a given cell tower can support before
becoming overloaded with and without our solution.  This data can be
extrapolated to predict the number of calls that a cell tower or phone box can
support with our solution in an emergency, thus providing a metric to gauge
social impact (i.e. the number of people that would be able to complete a call
with this solution in place and what types of calls people make based on
statistics versus not being able to call at all).

\subsection*{Partners}
We're talking to a professor from the International Computer Science Institute
(ICSI) about ad-hoc infrastructure and hoping to partner with them to use their
private cell towers for testing. We're also thinking of partnering with AT\&T's
Innovation Lab as a partnership with a telco would enable testing of our
technology on a much larger scale.

\subsection*{Risks and Constraints}
The primary risk is that large, bureaucratic telecommunications companies won't
be willing to adopt our technology because, even if funding for the actual
hardware itself were to come from other organizations, said companies wouldn't
want to add and maintain features to their existing infrastructure, especially
given the relatively niche use case for our technology (i.e. a large-scale
emergency).

\section{Team Roles}
Our product requires an in-depth knowledge of networking and software
engineering; we need to create a pipeline that will convert what is essentially
raw audio on-the-wire to a digital format, encapsulate it, and multiplex it back
on-the-wire.  Our team is comprised of all engineers, so we are in the fortunate
position where we all can and will be contributing to the codebase for this
project.

\begin{center}
    \begin{tabular}{ c | c | c }
        \textbf{Name} & \textbf{``Role(s)"} & \textbf{Areas of Expertise}\\
        \hline
        Mei & Product Manager and Engineer & SE, Networking\\
        Gavin & Business and Engineer & SE\\
        Michael & Engineer & SE\\
        Calvin & Engineer & SE\\
        Nick & Product Designer and Engineer & Networking, Security, Backend SE, and Hardware\\
    \end{tabular}
\end{center}

\section{Appendix}
\end{document}
