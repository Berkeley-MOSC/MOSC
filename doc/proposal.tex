\documentclass[10pt]{article}
\setlength\parindent{0pt}
\usepackage[margin=1in]{geometry}
\begin{document}
\title{\textsc{IEOR 185: Solution Proposal}}
\author{
    Gavin Chan,
    Michael Deng,
    Nicholas Fong,
    Calvin Lui,
    Mei Wan\\
    mosc[at]lists.berkeley.edu\\
}
\date{18 February 2016}
\maketitle

\section{Overview and Problem Statement}
\subsection*{Value Proposition}
During emergencies, phone networks often become unavailable with the influx of
calls.  This is because the phone infrastructure cannot support the influx of
connections, as it is designed to be resilient under common use, where a
minority of users are placing calls and sending texts at a given time.  By
multiplexing calls, network capacity can be increased, ensuring that civilians
and first-responders alike can stay in touch and effectively coordinate
response.
\subsection*{Target Market}
Telcos, local and federal emergency response agencies, municipal governments of
disaster prone regions around the globe.
\subsection*{Beneficiaries}
Everyone in the world after a disaster because our product enables more stable
communication.
\subsection*{Market Alternatives}
\subsubsection*{goTenna}\footnote{Snippet taken from Nick's ecosystem writeup}
The premise behind the goTenna is simple: a user with a goTenna connects their
phone to the goTenna via BTLE, and then uses a proprietary application to send
texts or a location to another goTenna user.   The goTenna has a range of about
a mile in urban areas, and up to four miles in open areas.  The goTenna is
marketed toward those that want to be able to communicate off-the-grid; goTenna
calls itself “adventure-ready” – combined with the images of backpacking and
the wilderness – indicating its target users are adventure-seekers whose
travels often take them away from cell phone towers.

\subsubsection*{Globalstar}
Globalstar is a satellite startup that was previously negotiating with the FCC
about licensing their own private frequency channel, so that they can expand
during times of crisis. It is currently unknown whether or not they were able
to attain this private licensed frequency.
\subsubsection*{???}

\section{MVP Solution Recommendation}
TODO

\subsection*{User Stories}
\begin{enumerate}
    \item After working for an extended period of time in an area devastated by
        an earthquake, a deployed USAR worker's radio dies.  He finds a group of
        Immediates in a collapsed building, and instead of having to run back to
        the command center to get help because the phone networks are down, he
        can call the IC and get help immediately
    \item After an earthquake, a mother is able to call her son's school and
        confirm that her son is safe
    \item TODO NICK
\end{enumerate}
\subsection*{Mockups}
TODO Nick

\subsection*{Future Roadmap}
We plan to expand this service to other disaster prone regions around the world.
This will include making deals and partnerships with the governments and telecom
companies there.  Mobile Mesh - if possible, we plan to implement hardware
solutions that will allow people to communicate via radio frequencies without
wifi/cell towers.

\section{Implementation}
\subsection*{KPIs}
How many cell phones can successfully get their calls and video calls through
multiplexing.

\subsection*{Partners}
We're talking to a professor from the International Computer Science Institute
(ICSI) about ad hoc infrastructure and hoping to partner with them to borrow
their cell towers for testing. We're also thinking of partnering with AT\&T
Innovation Lab as they might let us test our technology on a much larger scale.

\subsection*{Risks and Constraints}
The risk is that large companies like AT\&T and Verizon won't be willing to
adopt our technology because we're just college students (not the most
experienced ones either) working on a piece of software. Without partnership
with these companies, it's pretty hard to roll out our technology.

\section{Team Roles}
Mei -- Product Manager
Gavin -- Business
Michael -- Software Engineer
Calvin -- Software Engineer
Nick -- Software Engineer

All of us are EECS/CS students so we're all going to end up coding portions of
the project.

\section{Appendix}
\end{document}
